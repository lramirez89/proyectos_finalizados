\documentclass[10pt,a4paper]{article}
\usepackage[utf8]{inputenc}
\usepackage[spanish]{babel}
\usepackage{amsmath}
\usepackage{amsfonts}
\usepackage{amssymb}
\usepackage{multicol}
\title{Prática 1}
\date{}
\usepackage[top=2cm,bottom=2cm,left=1.5cm,right=1.5cm]{geometry}
\begin{document}
\maketitle

\section*{Parte 1 – Lógica proposicional}
\subsection*{Ejercicio 1}
Sean $p$ y $q$ variables proposicionales. ¿Cuáles de las siguientes expresiones son fórmulas bien formadas?
\begin{enumerate}

\item $(p\neg q)$ \textbf{NO.} Falta conector lógico entre $p$ y $q$
\item $p \vee q \wedge True$ \textbf{No.}Falta paréntesis.
\item $(p\rightarrow \neg p \rightarrow q)$  \textbf{No.} Faltan paréntesis
\item $\neg(p)$ \textbf{No.} No entra en ninguna de las reglas descritas: no es True ó False, no es una variable proposicional a secas. P es una FBF, pero la sintaxis no es la que corresponde según la tercera regla (en todo caso tendría que ser $\neg p$)
\item $(p \vee \neg p \wedge q)$ \textbf{No.} Faltan paréntesis.
\item $(True \wedge True \wedge True \wedge \ldots $ \textbf{No.}Falta paréntesis que cierren, faltan paréntesis entre conectores lógicos.
\item $(\neg p)$ \textbf{No.}$p$ es una FBP, por lo tanto $\neg p$ es una FBF. Pero los paréntesis de afuera están de más.
\item $(p \vee False)$ \textbf{Si.}$p$ es una variable proposicional, entoces es FBF. False lo es por la primera regla. La sintaxis cae dentro de la cuarta regla: dos FBF unidas con un conector lógico y cerradas por paréntesis.
\item $(p=q)$ \textbf{No.} = no es un conector lógico.

\end{enumerate}

\subsection*{Ejercicio 2}
Determine el valor de verdad de las siguientes fórmulas.
\begin{enumerate}
\item[a)] $(\neg a \vee b)$
	\begin{enumerate}
	\item[1.] $(\neg a \vee b)\equiv (\neg True \vee True)\leftrightarrow (False \vee True)\leftrightarrow True$
	\item[2.]$(\neg a \vee b)\equiv (\neg False \vee False)\leftrightarrow (True \vee False)\leftrightarrow True$
	\end{enumerate}

%\begin{tabular}{| c | c | c | c |}
%\hline
%$a$ & $b$ & $\neg a$ & $\neg a \vee b$\\ \hline
%True & True & False & True\\
%True & False & False & False\\
%False & True & True & True\\ 
%False & False & True & True\\ \hline
%\end{tabular} 

\item[b)]$((c \vee (y \wedge x))\vee b)$
	\begin{enumerate}
	\item[1.]$((c \vee (y \wedge x))\vee b)\equiv ((True \vee (False \wedge False))\vee True)\leftrightarrow ((True \vee False)\vee True)\leftrightarrow (True \vee True)\leftrightarrow True$
	\item[2.]$((c \vee (y \wedge x))\vee b)\equiv ((False \vee (True \wedge True))\vee False)\leftrightarrow ((False \vee True)\vee False)\leftrightarrow (True\vee False)\leftrightarrow True$ 
	\end{enumerate}
	
\item[c)]$\neg(c\vee y)$
	\begin{enumerate}
	\item[1.]$\neg(c\vee y)\equiv \neg(True\vee False)\leftrightarrow \neg True\leftrightarrow False$
	\item[2.]$\neg(c\vee y)\equiv \neg(False\vee True)\leftrightarrow \neg True\leftrightarrow False$ 
	\end{enumerate}
	
\item[d)]$(\neg(c\vee y)\leftrightarrow (\neg c \wedge \neg y))$
	\begin{enumerate}
	\item[1.]$(\neg(c\vee y)\leftrightarrow (\neg c \wedge \neg y))\equiv (\neg(True\vee False)\leftrightarrow (\neg True \wedge \neg False)) \\ 
	\leftrightarrow \\ 
	(\neg True\leftrightarrow (False \wedge True))\\
	\leftrightarrow \\
	(False\leftrightarrow False)\\
	\leftrightarrow\\
	True$
	\item[2.]$(\neg(c\vee y)\leftrightarrow (\neg c \wedge \neg y))\equiv (\neg(False\vee True)\leftrightarrow (\neg False \wedge \neg True)) \\ 
	\leftrightarrow \\ 
	(\neg True\leftrightarrow (True \wedge False))\\
	\leftrightarrow \\
	(False\leftrightarrow False)\\
	\leftrightarrow\\
	True$
	\end{enumerate}
	
\item[e)]$((c\vee y)\wedge (x\vee b))$
	\begin{enumerate}
	\item[1.]$((c\vee y)\wedge (x\vee b))\equiv ((True\vee False)\wedge (False\vee True)) \leftrightarrow (True\wedge True)\leftrightarrow True$
	\item[2.]$((c\vee y)\wedge (x\vee b))\equiv ((False\vee True)\wedge (True\vee False)) \leftrightarrow (True\wedge True)\leftrightarrow True$
	\end{enumerate}
	
\item[f)]$(((c \vee y) \wedge (x \vee b)) \leftrightarrow ((c \vee (y \wedge x)) \vee b))$
	\begin{enumerate}
	\item[1.]$(((c \vee y) \wedge (x \vee b)) \leftrightarrow ((c \vee (y \wedge x)) \vee b))\equiv (((True \vee False) \wedge (False \vee True)) \leftrightarrow ((True \vee (False \wedge False)) \vee True))\\
	\leftrightarrow\\
	((True \wedge True) \leftrightarrow ((True \vee False) \vee True))\\
	\leftrightarrow\\
	(True \leftrightarrow (True \vee True))\\
	\leftrightarrow\\
	(True \leftrightarrow True)\\
	\leftrightarrow\\
	 True$
	 \item[2.]$(((c \vee y) \wedge (x \vee b)) \leftrightarrow ((c \vee (y \wedge x)) \vee b))\equiv (((False \vee True) \wedge (True \vee False)) \leftrightarrow ((False \vee (True \wedge True)) \vee False))\\
	\leftrightarrow\\
	((True \wedge True) \leftrightarrow ((False \vee True) \vee False))\\
	\leftrightarrow\\
	(True \leftrightarrow (True \vee False))\\
	\leftrightarrow\\
	(True \leftrightarrow True)\\
	\leftrightarrow\\
	 True$
	\end{enumerate}

\item[g)]$(\neg c \wedge \neg y)$
	\begin{enumerate}
	\item[1.]$(\neg c \wedge \neg y)\equiv (\neg True \wedge \neg False)\leftrightarrow (False \wedge True)\leftrightarrow False$
	\item[2.]$(\neg c \wedge \neg y)\equiv (\neg False \wedge \neg True)\leftrightarrow (True \wedge False)\leftrightarrow False$
	\end{enumerate}
	
\item[h)] $((\neg c \rightarrow x) \wedge (y \vee (c \leftrightarrow \neg a)))$
	\begin{enumerate}
	\item[1.]$((\neg c \rightarrow x) \wedge (y \vee (c \leftrightarrow \neg a)))\equiv ((\neg True \rightarrow False) \wedge (False \vee (True \leftrightarrow \neg True)))\\
	\leftrightarrow\\
	((False \rightarrow False) \wedge (False \vee (True \leftrightarrow False)))\\
	\leftrightarrow\\
	(True \wedge (False \vee False))\\
	\leftrightarrow\\
	(True \wedge False)\\
	\leftrightarrow\\
	False$
	\item[2.]$((\neg c \rightarrow x) \wedge (y \vee (c \leftrightarrow \neg a)))\equiv ((\neg False \rightarrow True) \wedge (True \vee (False \leftrightarrow \neg False)))\\
	\leftrightarrow\\
	((True \rightarrow True) \wedge (True \vee (False \leftrightarrow True)))\\
	\leftrightarrow\\
	(True \wedge (True \vee False))\\
	\leftrightarrow\\
	(True \wedge True)\\
	\leftrightarrow\\
	True$
	\end{enumerate}
\end{enumerate}

\subsection*{Ejercicio 3}
Determine, utilizando tablas de verdad, si las siguientes fórmulas son tautologías,contradicciones o contingencias.

\begin{enumerate}
\item[a)] $(p\vee \neg p)$ \textbf{TAUTOLOGIA}

\begin{tabular}{| c | c | c |}
\hline
$p$ & $\neg p$ & $p \vee \neg p$\\ \hline
True & False & True\\
False & True & True\\ \hline
\end{tabular}

\item[b)]$(p \wedge \neg p)$ \textbf{CONTRADICCIÓN}

\begin{tabular}{| c | c | c |}
\hline
$p$ & $\neg p$ & $p \wedge \neg p$\\ \hline
True & False & False\\
False & True & False\\ \hline
\end{tabular}

\item[c)]$((\neg p \vee q) \leftrightarrow (p \rightarrow q))$ \textbf{TAUTOLOGIA}

\begin{tabular}{| c | c | c | c | c | c |}
\hline
$p$ & $q$ & $\neg p$ & $\neg p \vee q$ & $p \rightarrow q$ & $((\neg p \vee q) \leftrightarrow (p \rightarrow q))$ \\ \hline
True & True & False & True & True & True\\
True & False & False & False & False & True \\
False & True & True & True & True & True\\
False & False & True & True & True & True\\ \hline
\end{tabular}

\item[d)]$((p \vee q) \rightarrow p)$ \textbf{CONTINGENCIA}

\begin{tabular}{|c|c|c|c|}
\hline
$p$ & $q$ & $p \vee q$ & $((p \vee q) \rightarrow p)$ \\ \hline
True  & True  & True  & True \\ 
True  & False & True  & True \\
False & True  & True  & False \\
False & False & False & True \\ \hline

\end{tabular}

\item[e)]$(\neg(p \wedge q) \leftrightarrow (\neg p \vee q))$ \textbf{CONTINGENCIA}

\begin{tabular}{|c|c|c|c|c|c|c|}
\hline
$p$ & $q$ & $p \wedge q$ & $\neg(p \wedge q)$ & $\neg p$ & $\neg p \vee q$ &  $(\neg(p \wedge q) \leftrightarrow (\neg p \vee q))$\\ \hline
True  & True  & True  & False & False & True  & False \\
True  & False & False & True  & False & False & False \\
False & True  & False & True  & True  & True  & True \\
False & False & False & True  & True  & True  & True \\ \hline
\end{tabular}

\item[f)]$(\neg p \wedge q) \leftrightarrow (\neg p \vee \neg q))$ \textbf{CONTINGENCIA}

\begin{tabular}{|c|c|c|c|c|c|c|}
\hline
$p$ & $q$ & $\neg p$ & $\neg q$ & $\neg p \wedge q$ & $\neg p \vee \neg q$ &  $(\neg p \vee q) \leftrightarrow (\neg p \vee \neg q))$\\ \hline
True  & True  & False & False & False & False & True \\
True  & False & False & True  & False & True  & False \\
False & True  & True  & False & True  & True  & True \\
False & False & True  & True  & False & True  & False \\ \hline
\end{tabular}

\item[g)]$(p \rightarrow p)$ \textbf{TAUTOLOGIA}

\begin{tabular}{|c|c|}
\hline
$p$ & $p \rightarrow p$ \\ \hline
True  & True \\
False & True \\ \hline
\end{tabular}


\item[h)]$((p \wedge q) \rightarrow p)$ \textbf{TAUTOLOGIA}

\begin{tabular}{|c|c|c|c|}
\hline
$p$ & $q$ & $p \wedge q$ & $(p \wedge q) \rightarrow p$ \\ \hline
True  & True  & True  & True \\ 
True  & False & False & True \\
False & True  & False & True \\
False & False & False & True \\ \hline
\end{tabular} 

\item[i)]$((p \rightarrow (q \rightarrow r)) \rightarrow ((p \rightarrow q) \rightarrow (p \rightarrow r)))$ \textbf{TAUTOLOGIA}

\begin{tabular}{|c|c|c|c|c|c|c|c|c|}
\hline
$p$ & $q$ & $r$ & $q \rightarrow r$ & $(p \rightarrow (q \rightarrow r)$ & $p \rightarrow q$ & $p \rightarrow r$ & $(p \rightarrow q) \rightarrow (p \rightarrow r))$ & $((p \rightarrow (q \rightarrow r)) \rightarrow ((p \rightarrow q) \rightarrow (p \rightarrow r)))$ \\ \hline
T  & T & T & T & T & T & T & T & T\\
T  & T & F & F & F & T & F & F & T\\
T  & F & T & T & T & F & T & T & T\\
T  & F & F & T & T & F & F & T & T\\
F  & T & T & T & T & T & T & T & T\\
F  & T & F & F & T & T & T & T & T\\
F  & F & T & T & T & T & T & T & T\\
F  & F & F & T & T & T & T & T & T\\ \hline
\end{tabular}

\item[j)]$((p \wedge (q \vee r)) \leftrightarrow ((p \wedge q) \vee (p \wedge r)))$ \textbf{TAUTOLOGIA}

\begin{tabular}{|c|c|c|c|c|c|c|c|c|}
\hline
$p$ & $q$ & $r$ & $q \vee r$ & $p \wedge (q \vee r)$ & $p \wedge q$ & $p \wedge r$ & $(p \wedge q) \vee (p \wedge r)$ & $((p \wedge (q \vee r)) \leftrightarrow ((p \wedge q) \vee (p \wedge r)))$ \\ \hline
T  & T & T & T & T & T & T & T & T\\
T  & T & F & T & T & T & F & T & T\\
T  & F & T & T & T & F & T & T & T\\
T  & F & F & F & F & F & F & F & T\\
F  & T & T & T & F & F & F & F & T\\
F  & T & F & T & F & F & F & F & T\\
F  & F & T & T & F & F & F & F & T\\
F  & F & F & F & F & F & F & F & T\\ \hline
\end{tabular}

\end{enumerate}

\subsection*{Ejercicio 4}
Decimos que un conectivo es \textit{expresable} mediante otros si es posible escribir una fórmula utilizando exclusivamente estos últimos y que tenga la misma tabla de verdad que el primero (es decir, son equivalentes). Por ejemplo, la disyunción es expresable mediante la conjunción más la negación, ya que ($p \vee q$) tiene la misma tabla de verdad que $\neg(\neg p \wedge \neg q)$.

Muestre que cualquier fórmula de la lógica proposicional que utilice los conectivos $\neg$ (negación), $\wedge$ (conjunción), $\vee$ (disyunción), $\rightarrow$ (implicación), $\leftrightarrow$ (equivalencia) puede escribirse utilizando solo los
conectivos $\neg$ y $\vee$.

\begin{tabular}{|c|c|c|c|c|c|c|}
\hline
$p$ & $q$ & $\neg p$ & $\neg q$ & $\neg p \vee \neg q$ & $\neg(\neg p \vee \neg q)$ & $p \wedge q$ \\ \hline
T & T & F & F & F & T & T \\
T & F & F & T & T & F & F \\
F & T & T & F & T & F & F \\
F & F & T & T & T & F & F \\ \hline
\end{tabular}

\begin{tabular}{|c|c|c|c|c|}
\hline
$p$ & $q$ & $\neg p$ & $\neg p \vee q$  & $p \rightarrow q$ \\ \hline
T & T & F & T & T \\
T & F & F & F & F \\
F & T & T & T & T \\
F & F & T & T & T  \\ \hline
\end{tabular}

\begin{tabular}{|c|c|c|c|c|c|}
\hline
$p$ & $q$ & $p \rightarrow q$ & $q \rightarrow p$  & $(p \rightarrow q)\wedge (q\rightarrow p)$ & $p \leftrightarrow q$ \\ \hline
T & T & T & T & T & T \\
T & F & F & T & F & F\\
F & T & T & F & F & F\\
F & F & T & T & T & T \\ \hline
\end{tabular}

Como $\rightarrow$ es expresable mediante $\neg$ y $\vee$ y $\leftrightarrow$ es expresable mediante $\rightarrow$ entonces $\leftrightarrow$ es expresable mediante $\neg$ y $\vee$.

\subsection*{Ejercicio 5}
Sean las variables proposicionales $f$ , $e$ y $m$ con los siguientes significados:

$f \equiv$"es fin de semana"
$e \equiv$"Juan estudia"
$m \equiv$"Juan escucha música"
\begin{enumerate}
\item Escribir usando lógica proposicional las siguientes oraciones:

\begin{itemize}
\item “Si es fin de semana, Juan estudia o escucha música, pero no ambas cosas”

$((f \rightarrow e \wedge \neg m) \vee (f\rightarrow \neg e \wedge m))$

\item “Si no es fin de semana entonces Juan no estudia”

$(\neg f \rightarrow \neg e)$

\item “Cuando Juan estudia los fines de semana, lo hace escuchando música”

$((e \wedge f) \rightarrow m)$

\end{itemize}

\item Suponiendo que valen las tres proposiciones anteriores ¿se puede deducir que Juan no estudia? Justificar usando argumentos de la lógica proposicional.

Juan no estudia ya que $(\neg f \rightarrow \neg e)$

Ahora tomando el caso $f=True$ tenemos que $(f \rightarrow e \wedge \neg m)$, pero $((e \wedge f) \rightarrow m)$. Entonces se tiene que cumplir simultáneamente $m$ y $\neg m$ por lo que se llega a una contradicción.

\end{enumerate}

\section*{Parte 2 – Semántica de cortocircuito}
\subsection*{Ejercicio 6}
Asigne un valor de verdad (verdadero, falso o indefinido) a cada una de las siguientes expresiones lógicas, sabiendo que la proposición $p$ es verdadera, mientras que $q$ es falsa y $r$ está indefinida.

\begin{enumerate}
\item[a)] $((9 \leq 9) \wedge p)$

$((9 \leq 9) \wedge True) \equiv True \wedge True \equiv True$

\item[b)] $((3 \leq 2) \rightarrow (p \wedge q))$

$((3 \leq 2) \rightarrow (p \wedge q)) \equiv (False \rightarrow (True \wedge False)) \equiv True$

\item[c)]$((3 < 4) \rightarrow ((3 \leq 4) \vee r)$

$((3 < 4) \rightarrow ((3 \leq 4) \vee r) \equiv (True \rightarrow (True \vee \perp)) \equiv (True \rightarrow True) \equiv True$

\item[d)] $((3 > 9) \vee (r \wedge (q \wedge p)))$

$((3 > 9) \vee (r \wedge (q \wedge p))) \equiv (False \vee (\perp \wedge (False \wedge True))) \equiv (False \vee (\perp \wedge False)) \equiv (False \vee \perp) \equiv \perp$

\item[e)]$((p \wedge q) \wedge r)$

$((p \wedge q) \wedge r) \equiv (False \wedge \perp) \equiv False$

\item[f)]$((p \vee q) \vee r)$

$((p \vee q) \vee r) \equiv (True \vee \perp) \equiv True$

\item[g)] $((p \wedge r) \wedge ((q \rightarrow p) \vee (p \rightarrow (q \wedge r)))$

$((p \wedge r) \wedge ((q \rightarrow p) \vee (p \rightarrow (q \wedge r)))\equiv \\
((True \wedge \perp) \wedge ((False \rightarrow True) \vee (True \rightarrow (False \wedge \perp)))\equiv \\
(\perp \wedge (True \vee (True \rightarrow False)))\equiv \\
(\perp \wedge (True \vee False))\equiv \\
(\perp \wedge True)\equiv \perp$

\item[h)] $((p \wedge \neg q) \rightarrow (1 = 0))$\\
$((True \wedge \neg False) \rightarrow False)\equiv \\
((True \wedge True) \rightarrow False)\equiv (True \rightarrow False) \equiv False$

\item[i)] $(p \wedge ((5 - 7 + 3 = 0) \leftrightarrow (2^{2} - 1 > 3)))$

$(True \wedge (False \leftrightarrow False))\equiv \\
(True \wedge True)\equiv True$

\item[j)]$(\neg(p \vee r) \rightarrow r)$

$(\neg(True \vee \perp) \rightarrow \perp) \equiv \\
(\neg True \rightarrow \perp) \equiv \\
(False \rightarrow \perp)\equiv True$

\item[k)]$((p \rightarrow (1 > \log_{2} 0)) \leftrightarrow (2^{2} = 4 \wedge (p \wedge \neg q)))$

$((True \rightarrow (1 > \perp)) \leftrightarrow (True \wedge (True \wedge \neg False)))\equiv \\
((True \rightarrow \perp) \leftrightarrow (True \wedge (True \wedge True)))\equiv \\
(\perp \leftrightarrow (True \wedge True))\equiv \\
(\perp \leftrightarrow True)\equiv \perp$

\item[l)]$((p \rightarrow (q \rightarrow r)) \rightarrow ((p \rightarrow q) \rightarrow (p \rightarrow r)))$

$((True \rightarrow (False \rightarrow \perp)) \rightarrow ((True \rightarrow False) \rightarrow (True \rightarrow \perp)))\equiv \\
((True \rightarrow True) \rightarrow (False \rightarrow \perp))\equiv \\
(True \rightarrow True)\equiv True$

\end{enumerate}

\subsection*{Ejercicio 7}
\begin{enumerate}
\item[1.] Suponiendo que $p$ y $q$ no se encuentran indefinidas, simplificar las siguientes fórmulas:
	\begin{enumerate}
	\item[a)]$((p \wedge p) \wedge p)\equiv \\
	((p \wedge p)\equiv p$
	
	\item[b)]$(((p \wedge (\neg p \vee q)) \vee q) \vee (p \wedge (p \vee q)))\equiv \\
	((((p \wedge \neg p)\vee(p \wedge q)) \vee q) \vee (p \wedge (p \vee q)))  )\equiv \\
	(((False \vee(p \wedge q)) \vee q) \vee (p \wedge (p \vee q))  )\equiv \\
	(((p \wedge q) \vee q) \vee (p \wedge (p \vee q))  )\equiv \\
	((((p \vee q)\wedge (q \vee q)) \vee (p \wedge (p \vee q))  )\equiv \\
	(((p \vee q) \vee (p \wedge (p \vee q))  )\equiv \\
	(((p \vee q)\vee p) \wedge ((p \vee q)\vee (p \vee q)))\equiv \\
	((p \vee q) \wedge (p \vee q))\equiv (p\vee q)$
	
	\item[c)]$(\neg p \rightarrow \neg(p \rightarrow \neg q))\equiv \\
	(\neg p \rightarrow \neg(\neg p \vee q))\equiv \\
	(\neg p \rightarrow (p \wedge \neg q))\equiv \\
	(p \vee (p \wedge \neg q))\equiv \\
	((p \vee p) \wedge (p \vee \neg q))\equiv \\
	(p \wedge (p \vee \neg q))\equiv p$
	
	\item[d)]$(\neg((\neg(p \wedge q) \vee p \vee q) \rightarrow (\neg \neg p \vee \neg p)))\equiv \\
	(\neg((\neg p \vee \neg q) \vee p \vee q) \rightarrow (p \vee \neg p)))\equiv \\
	(\neg((\neg p \vee p)\vee (\neg q \vee q)) \rightarrow (p \vee \neg p)))\equiv \\
	(\neg(True \vee True) \rightarrow (p \vee \neg p)))\equiv \\
	(\neg True \rightarrow (p \vee \neg p)))\equiv \\
	(\neg( \neg True \vee (p \vee \neg p)))\equiv \\
	(\neg( False \vee (p \vee \neg p)))\equiv \\
	(\neg(p \vee \neg p))\equiv \neg True \equiv False$
	
	\item[e)]$(((p \rightarrow q) \vee (p \wedge \neg q)) \rightarrow q)\equiv \\
	(((\neg p \vee q) \vee (p \wedge \neg q)) \rightarrow q)\equiv \\
	((\neg (p \wedge \neg q) \vee (p \wedge \neg q)) \rightarrow q)\equiv \\
	(True \rightarrow q)\equiv q$	
	\end{enumerate}
\item[2.]¿Cuáles de las reglas de simplificación anteriores siguen valiendo cuando admitimos que $\alpha$,  $\beta$ y $\gamma$ puedan indefinirse?

False es absorbente para $\wedge$, True es absorbente para $\vee$.

\end{enumerate}
 
\section*{Parte 3 – Lógica de primer orden}
\subsection*{Ejercicio 8}
Sabiendo que $P (x)$ y $Q(x)$ son predicados unarios de $\mathbb Z \rightarrow \mathbb Z$ y $R(x, y)$ es un predicado binario de $\mathbb Z \times \mathbb Z \rightarrow \mathbb Z$, indicar en las siguientes fórmulas cuáles son $variables libres$ y cuáles $ligadas$.
\begin{enumerate}
\item[a)]$P (x)$ 

x libre
\item[b)]$(\forall y : \mathbb{Z})P (y)$ 

y ligada

\item[c)]$(\forall x : \mathbb{Z})P (y)$

y libre x ligada

\item[d)]$(P (\underbrace{x}_{libre}) \vee (\exists x : \mathbb{Z})P (\underbrace{x}_{ligada}))$

\item[e)]$(\exists x : \mathbb{Z})(P (x) \rightarrow Q(x))$

x ligada

\item[f)]$R(x, y)$

x e y libres

\item[g)]$(P (\underbrace{x}_{libre}) \vee (\exists x : \mathbb{Z})R(\underbrace{x}_{ligada}, \overbrace{y}^{libre})) \vee Q(\underbrace{x}_{libre})$

\item[h)]$\underbrace{(\exists x : \mathbb{Z})\underbrace{(\forall y : \mathbb{Z})R(x, y)}_{y\ ligada, x\ libre}}_{x,y\ ligadas}$

\item[i)]$(\forall x : \mathbb{Z})(P (x) \wedge Q(x))$

x ligada

\item[j)]$(\forall x : \mathbb{Z})(P (y) \wedge Q(x))$

x ligada, y libre

\item[k)]$\underbrace{(\forall z : \mathbb{Z})(P (z) \rightarrow (\underbrace{(\exists x : \mathbb{Z})(P (x) \wedge R(x, z)))}_{z\ libre,x\ ligada})}_{x,y,z\ ligadas}$

\item[l)]$\underbrace{(\forall z : \mathbb{Z})(P (z) \rightarrow (\underbrace{ (\exists x : \mathbb{Z})(P (z) \wedge R(x, z)))}_{x\ ligada,z\ libre})}_{x,y,z\ ligadas}$
\end{enumerate}

\subsection*{Ejercicio 9}
Dadas $w$, $x$, $y$, $z$ : $\mathbb{Z}$ y sabiendo que $w = 0$, $x = 1$, $y = 3$ y $z = 4$, determinar el valor de verdad de las siguientes fórmulas en $\mathbb{Z}$.
\begin{enumerate}
\item[a)]$x = y$

$\equiv 1=3 \equiv False$

\item[b)]$x + y = z$

$\equiv 1+3=4 \equiv True$

\item[c)]$(\forall x : \mathbb{Z})(\exists y : \mathbb{Z})(x = y)$

Verdadero. Sea $x \in \mathbb{Z}$, elijo $y=x$

\item[d)]$(\exists y : \mathbb{Z})(\forall x : \mathbb{Z})(x = y)$

Falso. Supongamos que $\exists y\in \mathbb{Z}$ que cumple lo pedido. Entonces si vale $x=y$ tiene que valer $x+1=y$ ya que $x+1\in \mathbb{Z}$ también. Pero $x+1=y \rightarrow \underbrace{x+1=x}_{ya\ que\ y=x}\rightarrow 1=0$

\item[e)]$(\forall x : \mathbb{Z})(x \geq y \rightarrow x \geq z)$

$(\forall x : \mathbb{Z})(x \geq 3 \rightarrow x \geq 4)$

Falso. Contraejemplo:$x=3$

\item[f)]$(\exists i : \mathbb{Z})(x \leq i \vee i \leq y)$

$(\exists i : \mathbb{Z})(1 \leq i \vee i \leq 3)$

Verdadero.Por ejemplo para $i=1$ vale.

\item[g)]$(\exists i : \mathbb{Z})(y \leq i \vee i \leq x)$

$(\exists i : \mathbb{Z})(3 \leq i \vee i \leq 1)$

Falso. No existe número menor que 1 y a la vez mayor que 3.

\item[h)]$(\forall i : \mathbb{Z})(x \leq i \vee i \leq y)$

$(\forall i : \mathbb{Z})(1 \leq i \vee i \leq 3)$

Falso. Contraejemplo $i=4$

\item[i)]$(\forall i : \mathbb{Z})(x \leq i \vee x = 0)$

$(\forall i : \mathbb{Z})(1 \leq i \vee 1 = 0)$

Falso
\end{enumerate}

\subsection*{Ejercicio 10}

Determinar el valor de verdad de cada una de las siguientes fórmulas. Cuando alguna no sea
verdadera, encontrar valores para las variables que la hagan falsa o la indefinan, cuando sea posible.
\begin{enumerate}
\item[a)]$(\forall x : \mathbb{R})(\exists y : \mathbb{R})(x \leq y)$

Verdadero. Sea $x\in \mathbb{R}$, si tomamos $y=x+1$ luego $x\leq y \leftrightarrow x\leq x+1 \leftrightarrow 0\leq 1$

\item[b)]$(\exists x : \mathbb{R})(\forall y : \mathbb{R})(x \leq y)$

Falso. Supongamos que existe $x$ que cumple $x \leq y, \forall y\in \mathbb{R}$. Pero si tomamos $y=x-1$ tenemos que $y<x\rightarrow x\not \leq y$

\item[c)]$(\forall y : \mathbb{R})(\forall x : \mathbb{R})(x \leq y)$

Falso. Contraejemplo: $x=2,y=1$. $\perp$ si $y$ está indefinido.

\item[d)]$(\forall x : \mathbb{R})(x \leq y)$

Falso.$y$ es una variable libre, pero independientemente de eso, $y$ no está acotadp superiormente. Contraejemplo: elijo $x=y+1$. $\perp$ si $y$ está indefinido.

\item[e)]$(\exists x : \mathbb{R})(x = y)$. $\perp$ si $y$ está indefinido.

Verdadero. Elijo $x=y$

\item[f)]$(\forall x : \mathbb{R})(x = y)$

Falso. Contraejemplo: $x=y+1$. $\perp$ si $y$ está indefinido.

\item[g)]$(\forall x, y : \mathbb{R})(y = 0 \vee (\exists z : \mathbb{R})(x/y = z))$

Verdadero. Si $y=0$ toda la expresión es verdaera y no se evalúa el $\exists$. Si $y\not =0$ entoces evualuamos $x/y = z$, que no se indefine porque $y$ no es cero y siemore hay un resutlado porque no hay otras restricciones para la división y $x/y  \in \mathbb{R}$

\item[h)]$(\forall x, y : \mathbb{R})(\exists z : \mathbb{R})(x \cdot y = z)$

Verdaero. La operación $\cdot$ no se indefine para ningún valor en $\mathbb{R}$ y además es cerrada para el producto, por lo tanto $x \cdot y \in \mathbb{R} $

\item[i)]$(\forall x, y : \mathbb{R})(\exists z : \mathbb{R})(x \cdot z = y)$

Si $y$ esstá indefinido, la expresión es $\perp$. Si $y=0$ es verdadero ($x$ puede valer cualquie valor y $z=0$). Si $y\not =0$ es falso porque en el caso $x=0$ tenemos que $x \cdot y = 0$

\item[j)]$(\forall x, y : \mathbb{R})(x \leq y \vee y \leq x)$

Verdadero: relación de orden total en $\mathbb{R}$

\item[k)]$(\forall x, y : \mathbb{R})(\exists z : \mathbb{R})(x < z \wedge z < y)$

Falso: falla cuando $y \leq x$.

\item[l)]$(\forall x, y : \mathbb{R})(x > y \rightarrow (\exists z : \mathbb{R})(x < z \wedge z < y))$

Falso. No existe un número mayor que el mayor($x$) y a la vez menor que el menor($y$)

\item[m)]$(\forall x, y : \mathbb{R})((\exists z : \mathbb{R})(x/y = z) \vee y = 0)$

$\perp$. Se indefine cuando $y=0$
\end{enumerate}

\section*{Parte 4 – Relaciones de fuerza}
\subsection*{Ejercicio 11}
Dadas las proposiciones lógicas $\alpha$ y $\beta$, se dice que $\alpha$ es más fuerte que $\beta$ si y solo si $\alpha \rightarrow \beta$ es una tautología. En este caso, también decimos que $\beta$ es más débil que $\alpha$. Determinar la relación de fuerza de los siguientes pares de fórmulas:
\begin{enumerate}
\item $True$, $False$

$False$ es más fuerte que $True$, porque $False \rightarrow True$ pero no es cierto que $True\rightarrow False$

\item $(p \wedge q), (p \vee q)$

$(p \wedge q)$ es más fuert que $(p \vee q)$

\begin{tabular}{|c|c|c|c|c|c|}
\hline
$p$ & $q$ & $p \wedge q$ & $p \vee q$ & $(p \wedge q)\rightarrow (p \vee q)$ & $(p \vee q)\rightarrow (p \wedge q)$ \\ \hline
T & T & T & T & T & T\\
T & F & F & T & T & F\\
F & T & F & T & T & F\\
F & F & F & F & T & T\\ \hline
\end{tabular}

\item $True$, $True$

Igual de fuertes (se implican el uno al otro)

\item $p$, $(p \wedge q)$

$(p \wedge q)$ es más fuerte, ya que para que el "y" sea verdadero ambos tienen que serlo. Luego $p$ es verdadero. $p$ no es más fuerte que $(p \wedge q)$, ya que $p$ puede ser verdaero, pero si $q$ es falso $(p \wedge q)$ es falso.

\item $False$, $False$

Son igual de fuertes

\item $p$, $(p \vee q)$

$p$ es más fuerte, ya que si $p$ es verdadero entonces $(p \vee q)$ lo es ya uqe basta con que uno de los dos sea verdadero. $(p \vee q)$ no es más fuerte (es más débil) ya que $(p \vee q)$ puede ser verdadero y $p$ falso (caso $p=False$ y $q=True$).

\item $p$, $q$

No se puede establecer una relación de fuerza. 

\item  $p$, $(p \rightarrow q)$

Ninguno es más fuerte. Si $p=True$ no es cierto que $p \rightarrow (p \rightarrow q)$ porque si $q$ es falso falla. También tenemos que $(p \rightarrow q)\not \rightarrow p$, ya que si $p=False$ el antecedente es verdadero y el consecuente falso.
\end{enumerate}
¿Cuál es la proposición más fuerte y cuál la más débil de las que aparecen en este ejercicio?

$False$ es la más fuerte y $True$ es la más débil.

\subsection*{Ejercicio 12}
Sea $x : \mathbb{Z}$, se tienen los siguientes predicados:
\begin{enumerate}
\item $P_{1}  \equiv \{(\exists k : \mathbb{Z})(2 \cdot k = x)\}$
\item $P_{2} \equiv \{x = 2\}$
\item $P_{3} \equiv \{(\exists k : \mathbb{Z})(2 \cdot k = x \vee 2 \cdot k + 1 = x)\}$
\end{enumerate}
Indicar la relación de fuerza entre ellos. Justifique su respuesta en \textbf{cada} caso:

$P_{1}$: "x es par", $P_{2}$: "x es 2", $P_{3}$: "x es par ó impar"
\begin{itemize}
\item $P_{2}$ es más fuerte que $P_{1}$: Si $x=2$ entonces es par. Pero no vale la recíproca, no todos los números pares son 2.
\item $P_{2}$ es más fuerte que $P_{3}$: Si $x=2$ entonces es par ó es impar. Pero no vale la recíproca.
\item $P_{1}$ es más fuerte que $P_{3}$: Si $x$ es par entonces es par o impar. Pero si x es par o impar no necesariamente es par (si $x$ llega a ser impar no vale $P_{1}$)
\end{itemize}

\subsection*{Ejercicio 13}
Sea $[a : \mathbb{Z}]$ y el predicado auxiliar $esPrimo(i : \mathbb{Z})$, se tienen los siguientes predicados:
\begin{enumerate}
\item $P_{1}\equiv \{(\exists i : \mathbb{Z}) (0 \leq i < |a| \wedge esPrimo(a[i]))\}$
\item $P_{2} \equiv \{(\forall i : \mathbb{Z}) (0 \leq i < |a| \rightarrow esPrimo(a[i]))\}$
\item 
$P_{3}\equiv \left\lbrace \left(\sum_{i=1}^{|a|-1}  \beta(esPrimo(a[i]))\right)=|a| \right\rbrace $
\end{enumerate}
Indicar la relación de fuerza entre ellos. Justifique su respuesta en \textbf{cada} caso.
\begin{itemize}
\item $P_{2}$ es más fuerte que $P_{1}$: que entre 0 y $a$ todos sean primos implica que existe al menos algún primo en $a$, pero no la inversa.
\item $P_{2}$ y $P_{3}$ son igual de fuertes: ambas expresiones dicen lo mismo. En $P_{3}$ si todos son primos estoy sumando 1 $|a|$ veces. Y la única forma de sumar $|a|$ en $P_{3}$ es que todos entre 0 y $|a|$ sean primos.
\item $P_{3}$ es más fuerte que $P_{1}$: que entre 0 y $a$ todos sean primos implica que existe al menos algún primo en $a$, pero no la inversa.
\end{itemize}

\subsection*{Ejercicio 14}
Sean $x, i : \mathbb{Z}$ y $[a : \mathbb{Z}]$, se tienen los siguientes predicados:
\begin{enumerate}
\item $P_{1} \equiv \{(\forall i : \mathbb{Z}) (1 \leq i < |a| \rightarrow a[i] \not = x) \wedge (a[0] = x)\}$
\item $P_{2} \equiv \{(\forall i : \mathbb{Z}) (1 \leq i < |a| \rightarrow a[i] \not = x) \vee (a[0] = x)\}$
\item $P_{3} \equiv \{(\forall i, j : \mathbb{Z}) (0 \leq i < j < |a| \rightarrow a[i] < a[j]) \wedge (a[0] = x)\}$
\end{enumerate}
Indicar la relación de fuerza entre ellos. Justifique su respuesta en \textbf{cada} caso.
\begin{itemize}
\item $p_{1}$ es más fuerte que $P_{2}$: caso $(p \wedge q)$ es más fuerte que $(p \vee q)$ (Ejercicio 11 2.)
\item No hay relación de fuerza entre $P_{1}$ y $_{3}$. Que entre 1 y $|a$ ninguno sea $x$ no implica que estén ordenados ($P_{1}\not \rightarrow P_{3}$), y que estén todos ordenados no implica que ninguno sea $x$ ($P_{3}\not \rightarrow P_{1}$).
\item $P_{3}$ es más fuerte que $P_{2}$. $P_{3}$ verdadero implica que se cumple $(a[0] = x)$, una de las dos proposiciones en la disyunción $P_{2}$. No vale la recíproca porque $(a[0] = x)$ no es necesariamente cierto en $P_{2}$ y no hay relacion de fuerza entre las primeras partes de la disyunción de $P_{2}$ y la conjunción de $P_{3}$
\end{itemize}

\subsection*{Ejercicio 15}
Sean $x, i : \mathbb{Z}$ y $[a : \mathbb{Z}]$, se tienen los siguientes predicados:
\begin{enumerate}
\item $P_{1} \equiv \{(\exists j : \mathbb{Z})(0 \leq j < |a| \wedge a[j] = x)\}$
\item $P_{2} \equiv \{0 \leq i < |a| \wedge a[i] = x\}$
\item $P_{3} \equiv \{(\forall j : \mathbb{Z})(0 \leq j < |a| - 1 \rightarrow a[j] \not = x) \wedge a[|a| - 1] = x\}$
\end{enumerate}
Indicar la relación de fuerza entre ellos. Justifique su respuesta en \textbf{cada} caso.
\begin{itemize}
\item $P_{2}$ es más fuerte que $P_{1}$. Si en particular $i$ cumple que $a[i]=x$ eso implica que existeal menos un número en el rago pedido que cumple $a[i]=x$. No vale la recíproca porque $i$ es uan variable libre. Si $i$ está indefinido o está fuera de rango todo se indefine.
\item No hay relación de fuerza entre $P_{1}$ y $P_{3}$. $P_{1}$ dice que entre 0 y $|a|$ todos son $x$, pero $P_{3}$ dice que entre 0 y $|a|-|$ ninguno es $x$ ($P_{1}\not \rightarrow P_{3}$). Por el mimsmo motivo no vale la recíproca. En el caso particular en el que $|a|=1$ no hay elementos entre 0 y $|a|$, ambas expresiones son iguales y ambas son igual de fuertes.
\item $P_{2}$ es más fuerte que $P_{3}$ si $i=|a|-1$. En caso contrario no hay relación de fuerza porque ambas expresiones son contradictorias.
\end{itemize}

\end{document}















