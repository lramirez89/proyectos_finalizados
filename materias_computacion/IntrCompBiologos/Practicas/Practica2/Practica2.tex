\documentclass[10pt,a4paper]{article}
\usepackage[utf8]{inputenc}
\usepackage{amsmath}
\usepackage[spanish]{babel}
\usepackage{amsfonts}
\usepackage{amssymb}
\title{Práctica 2 – Python básico}
\usepackage[top=2cm,bottom=2cm,left=1.5cm,right=1.5cm]{geometry}
\date{}

\begin{document}
\maketitle
\section*{Parte 1 – Problemas algebraicos}


\subsection*{Ejercicio 1}

Especifique e implemente las siguientes funciones $booleanas$:
\begin{enumerate}
\item[a)]\textbf{noEsCero(n)}: devuelve True si n es distinto de cero

\textbf{problema noEsCero}$(n : \mathbb{R})=res:Bool$\{ \\
	\textbf{  requiere:} $True$\\
	\textbf{  asegura:}$res == (n\not = 0)$\\
	\}


\item[b)]\textbf{iguales(n1, n2)}: devuelve True si n1 es igual a n2.

\textbf{problema iguales}$(n1,n2 : t)=res:Bool$\{ \\
	\textbf{  requiere:} $True$\\
	\textbf{  asegura:}$res == (n1 == n2))$\\
	\}

\item[c)]\textbf{menor(n1, n2)}: devuelve True si n1 es menor (estricto) a n2.

\textbf{problema menor}$(n1,n2 : t)=res:Bool$\{ \\
	\textbf{  requiere:} $True$\\
	\textbf{  asegura:}$res == (n1<n2) $\\
	\}

\item[d)]\textbf{par(n)}: devuelve True si n es un número par.

\textbf{problema par}$(n : \mathbb{Z})=res:Bool$\{ \\
	\textbf{  requiere:} $True$\\
	\textbf{  asegura:}$res == (n \mod 2==0)$\\
	\}

\item[e)]\textbf{divisible(n, d)}: devuelve True si n es divisible por d.

\textbf{problema divisible}$(n,d : \mathbb{Z})=res:Bool$\{ \\
	\textbf{  requiere:} $d\not = 0$\\
	\textbf{  asegura:}$res == (n \mod d==0)$\\
	\}

\item[f)]\textbf{imparDivisiblePorTresOCinco(n)}: devuelve True si n es divisible por 3 o por 5 pero no por 2.

\textbf{problema imparDivisiblePorTresOCinco}$(n : \mathbb{Z})=res:Bool$\{ \\
	\textbf{  requiere:} $True$\\
	\textbf{  asegura:}$res == (((n \mod 3==0)\vee (n \mod 5==0)) \wedge (n \mod 2 \not = 0))$\\
	\}

\end{enumerate}

\subsection*{Ejercicio 2}
Especifique e implemente las siguientes funciones sobre enteros:
\begin{enumerate}
\item[a)]\textbf{factorial(n)}: devuelve el valor del factorial de n.

\textbf{problema factorial}$(n : \mathbb{Z})=res:\mathbb{Z}$\{ \\
	\textbf{  requiere:} $n \geq 0$\\
	\textbf{  asegura:}$res == \prod_{i=1}^{n}i$\\
	\}

\item[b)]\textbf{sumaDivisores(n)}: devuelve la suma de todos los divisores positivos de n.

\textbf{problema sumaDivisores}$(n : \mathbb{Z})=res:\mathbb{Z}$\{ \\
	\textbf{  requiere:} $True$\\
	\textbf{  asegura:}$res == \sum_{i=1}^{|n|} \beta(n \mod i ==0)*i$\\
	\}
\item[c)]\textbf{primo(n)}: devuelve True si n es un número primo.

\textbf{problema primo}$(n : \mathbb{Z})=res:Bool$\{ \\
	\textbf{  requiere:} $True$\\
	\textbf{  asegura:}$esPrimo(n)$\\
	\}
	
	$esPrimo(n:\mathbb{Z})\equiv \{(n\not = 0 \rightarrow (res==(\forall k: \mathbb{Z})(1<k<|n|\rightarrow n \mod k \not = 0))) \wedge (n==0 \rightarrow res== False) \} $
\item[d)]\textbf{menorDivisiblePorTres(n)}: dado un n positivo, devuelve el menor número mayor a n tal que sea divisible por 3.

\textbf{problema menorDivisiblePorTres}$(n : \mathbb{Z})=res:\mathbb{Z}$\{ \\
	\textbf{  requiere:} $n>0$\\
	\textbf{  asegura:}$res \mod 3 ==0 \wedge res>n$\\
	\textbf{asegura:} $(\forall k:\mathbb{Z})(n<k<res \rightarrow k \mod 3\not = 0)$\\
	\}
\item[e)]\textbf{mayorPrimo(n1, n2)}: devuelve True si n1 es el mayor primo que divide a n2.

\textbf{problema mayorPrimo}$(n1,n2 : \mathbb{Z})=res:Bool$\{ \\
	\textbf{  requiere:} $True$\\
	\textbf{  asegura:}$res== (esPrimo(n1)\wedge (n2 \mod n1 ==0) \wedge (\forall k: \mathbb{Z})((n1<k\leq n2 \wedge esPrimo(k))\rightarrow n2 \mod k \not = 0)$\\
	\}

\item[f)]\textbf{mcd(n1, n2)}: devuelve el máximo común divisor entre n1 y n2.

\textbf{problema mcd}$(n1,n2 : \mathbb{Z})=res:\mathbb{Z}$\{ \\
	\textbf{  requiere:} $\neg(n1==0 \wedge n2==0)$\\
	\textbf{  asegura:}$(n1 \mod res ==0) \wedge (n2 \mod res==0)$\\
	\textbf{asegura:}$(\forall k:\mathbb{Z})(res<k\leq |n2|\rightarrow \neg(n1 \mod k == 0 \wedge n2\mod k ==0))$
	\}
\end{enumerate}

\section*{Parte 2 – Secuencias}
\subsection*{Ejercicio 3}
Especifique e implemente las siguientes funciones sobre secuencias. Para la implementación, en los
casos en los que exista una función equivalente en Python, no está permitido utilizarla
\begin{enumerate}
\item[a)]\textbf{suma(a)}: devuelve la suma de todos los elementos de la lista a

Siendo $\mathbb{T}$ un tipo que soporte la suma:

\textbf{problema suma}$([a : \mathbb{T}])=res:\mathbb{T}$\{ \\
	\textbf{  requiere:} $True$\\
	\textbf{  asegura:}$res==suma(a)$\\
	\}
	
	$suma([a:\mathbb{T}])\equiv \{\sum^{|a|-1}_{i=0}a[i] \} $
	
	¿Está bien que si la lista es vacía la suma de 0?
\item[b)]\textbf{promedio(a)}: devuelve el promedio de todos los elementos de la lista a. ¿Qué ocurre si a no tiene elementos?

Siendo $\mathbb{T}$ un tipo que soporte la suma y la división:

\textbf{problema promedio}$([a : \mathbb{T}])=res:\mathbb{T'}$\{ \\
	\textbf{  requiere:} $|a|>0$\\
	\textbf{  asegura:}$res==suma(a)/|a|$\\
	\}
	
	Si aceptáramos como entrada válida la lista vacía, el asegura se indefiniría.

\item[c)]\textbf{maximo(a)}: devuelve el máximo entre todos los elementos de la lista a.

\textbf{problema maximo}$([a : \mathbb{T}])=res:\mathbb{T}$\{ \\
	\textbf{  requiere:} $|a|>0$\\
	\textbf{  asegura:}$(\exists i:\mathbb{Z})(0\leq i<|a|\rightarrow a[i]== res)$\\
	\textbf{asegura:} $esCotaSuperior(res,a)$\\
	\}
	
	$esCotaSuperior(val:\mathbb{T},[a:\mathbb{T}])\equiv \{(\forall i:\mathbb{Z})(0\leq i<|a| \rightarrow val\geq a[i]) \} $
	
\item[d)]\textbf{listaDeAbs(a)}: devuelve una lista con los valores absolutos de cada elemento de la lista a.

\textbf{problema listaDeAbs}$([a : \mathbb{T}])=res:[\mathbb{T}]$\{ \\
	\textbf{  requiere:} $True$\\
	\textbf{asegura:}$(\forall i:\mathbb{Z})(0\leq i<|a|\rightarrow pertenece(|a[i]|,res))$\\
	\textbf{asegura:}$(\forall x:\mathbb{T})(cantApariciones(x,res)== cantApariciones(x,a)+ cantApariciones(-x,a))$\\
	\}
	
	$cantApariciones(val:\mathbb{T},[a:\mathbb{T}])\equiv \{\sum^{|a|-1}_{i=0}\beta(a[i]) \} $
	
	$pertenece(val:\mathbb{T},[a:\mathbb{T}])\equiv \{cantApariciones(val,a)\not = 0 \} $
	
\item[e)]\textbf{todosPares(a)}: devuelve True si todos los elementos de la lista a son pares.

\textbf{problema todosPares}$([a : \mathbb{Z}])=res:Bool$\{ \\
	\textbf{  requiere:} $True$\\
	\textbf{  asegura:}$res==(\forall i:\mathbb{Z})(0\leq i<|a|\rightarrow a[i]\mod 2 == 0)$\\
	\}
\item[f)]\textbf{maximoAbsoluto(a)}: devuelve el máximo entre los valores absolutos de todos los elementos de la lista a.

\textbf{problema maximoAbsoluto}$([a : \mathbb{T}])=res:\mathbb{T}$\{ \\
	\textbf{  requiere:} $|a|>0$\\
	\textbf{  asegura:}$(\exists i:\mathbb{Z})(0\leq i<|a|\rightarrow |a[i]|== res)$\\
	\textbf{asegura:} $(\forall i:\mathbb{Z})(0\leq i<|a| \rightarrow res\geq |a[i]|)$\\
	\}
	
\item[g)]\textbf{divisores(n)}: devuelve una lista con todos los divisores positivos de n

\textbf{problema divisores}$([a : \mathbb{Z}])=res:[\mathbb{Z}]$\{ \\
	\textbf{  requiere:} $n\not = 0$\\
	\textbf{asegura:}$(\forall k:\mathbb{Z})(1\leq k\leq |n| \wedge n \mod k ==0 \rightarrow pertenece(k,res))$\\
	\}
	
	Obs: $res$ tiene que tener todos los divisores positivos, pero no dice nada sobre elementos repetidos ni sobre contener números que no sean divisores.

Si además quiero que en $res$ no haya ningún número que no sea divisor agrego:

$$(\forall i: \mathbb{Z})(0\leq i < |res|\rightarrow n \mod res[i]==0)$$

Si además no quiero que haya elementos repetidos agrego:

$$(\forall i,j:\mathbb{Z})((0 \leq i <|res| \wedge 0\leq j<|res| \wedge i\not = j)\rightarrow res[i]\not = res[j])$$

	
\item[h)]\textbf{cantidadApariciones(a, x)}: devuelve la cantidad de veces que aparece el elemento x en la lista a.

\textbf{problema cantidadApariciones}$([a : \mathbb{T}],x:\mathbb{T})=res:\mathbb{Z}$\{ \\
	\textbf{  requiere:} $True$\\
	\textbf{asegura:}$res==cantApariciones(x,a)$\\
	\}
	
\item[i)]\textbf{masRepetido(a)}: devuelve el elemento que más veces aparece repetido en la lista a.

\textbf{problema masRepetido}$([a : \mathbb{T}])=res:\mathbb{T}$\{ \\
	\textbf{  requiere:} $|a|>0$\\
	\textbf{asegura:}$(\forall i:\mathbb{Z})(0\leq i<|a|\rightarrow cantApariciones(res,a)\geq cantApariciones(a[i],a))$\\
	\}
\item[j)]\textbf{ordenAscendente(a)}: devuelve True si todos los elementos de la lista a aparecen en orden ascendente. Ejemplos:
	\begin{itemize}
	\item ordenAscendente([]) == True
	\item ordenAscendente([1,2,4]) == True
	\item ordenAscendente([4,2,1]) == False
	\end{itemize}
	
	\textbf{problema ordenAscendennte}$([a : \mathbb{T}])=res:Bool$\{ \\
	\textbf{  requiere:} $True$\\
	\textbf{asegura:}$(\forall i:\mathbb{Z})(0\leq i<|a|-1\rightarrow a[i]<a[i+1])$\\
	\}
	
\item[k)]\textbf{reverso(a)}: devuelve una lista que cumple que sus elementos son los mismos que los de a, pero se encuentran en el orden inverso. Ejemplos:
	\begin{itemize}
	\item reverso([’h’,’o’,’l’,’a’]) == [’a’,’l’,’o’,’h’]
	\item reverso(reverso([’h’,’o’,’l’,’a’])) == [’h’,’o’,’l’,’a’]
	\end{itemize}
	
	\textbf{problema reverso}$([a : \mathbb{T}])=res:[\mathbb{T}]$\{ \\
	\textbf{  requiere:} $True$\\
	\textbf{asegura} $|res|==|a|$\\
	\textbf{asegura:}$(\forall i:\mathbb{Z})(0\leq i<|a|\rightarrow res[i]==a[|a|-1-i])$\\
	\}
\end{enumerate}


\subsection*{Ejercicio 4}
Implemente funciones en Python que cumplan con las siguientes especificaciones. Proponga un
nombre declarativo en castellano para cada función implementada.
\begin{enumerate}
\item[a)]
\textbf{problema A}$(n : \mathbb{Z})=x: \mathbb{R}$\{ \\
	\textbf{  requiere:} $n \geq0;$\\
	\textbf{  asegura:}$x^{2} == n;$\\
	\}
	
Nombre propuesto: raizCuadrada

Explicación de la implementación:
$$x= \sqrt[•]{a}$$
Podemos hallar la raiz cuadrada $x\in \mathbb{Z}$ de un numero entero $a$ hallando el $x\in \mathbb{Z}$ más grande tal $x*x\leq a $ mediante un ciclo \textbf{while}. El problema es que la especificación nos pide devolver $\mathbb{R}$. Lo más cercano a un $\mathbb{R}$ que podemos devolver es un \textit{float} con cierta cantidad de cifras después de la coma.

Notar que multiplicar un numero por diez es correr la coma un lugar a la derecha, entonces $\sqrt{a}\ast 10^{d}$ representa a la $\sqrt{a}$ con la coma corrida $d$ lugares a la derecha.

Si metemos a $10^{d}$ dentro de la raiz nos queda $\sqrt{a}\ast 10^{d}=\sqrt{a\ast 10^{2d}}$ entonces 
$$\sqrt{a}=\frac{\sqrt{a\ast 10^{2d}}}{10^{d}}$$

De este modo al hallar a $x'$, el entero más grande tal que $x'\ast x' \leq \sqrt{a\ast 10^{2d}}$ estaríamos hallando $\sqrt{a}\ast 10^{d}$. De este modo para obtener $\sqrt{a}$ solo faltaría dividir $x'$ por $10^{d}$.
\item[b)]

\textbf{problema B}$([a : \mathbb{Z}])=x: \mathbb{Z}$\{ \\
	\textbf{  requiere:} $True;$\\
	\textbf{  asegura:}$x==(\sum^{|a|-1}_{i=0}\beta (i \mod 2==0)\cdot a[i];$\\
	\}
	
Nombre propuesto: sumaPosPares

\item[c)]

\textbf{problema C}$([a : \mathbb{Z}])=b: \mathbb{B}$\{ \\
	\textbf{  requiere:} $True;$\\
	\textbf{  asegura:}$b==(\forall i:\mathbb{Z})(0\leq i <|a|\rightarrow (a[i]==a[|a|-1-i]));$\\
	\}

Nombre propuesto: capicua	
	
\item[d)]

\textbf{problema D}$([a : \mathbb{Z}])=r: \mathbb{Z}$\{ \\
	\textbf{  requiere:} $|a|>0;$\\
	\textbf{  asegura:}$r== (\sum^{|a|-1}_{i=0}\beta (i \mod 2 ==1)\cdot a[i])/(\frac{|a|}{2});$\\
	\}
	
Nombre propuesto: promedioImpares
	
\item[e)]

\textbf{problema E}$([a : \mathbb{Z}])=r: \mathbb{Z}$\{ \\
	\textbf{  requiere:} $|a|>0;$\\
	\textbf{  asegura:}$(\exists i: \mathbb{Z})(0\leq i \wedge i<|a| \wedge (\forall j:\mathbb{Z})(0\leq j \wedge j<|a| \rightarrow a[i]\leq a[j])\wedge r==a[i]);$\\
	\}	
	
Nombre propuesto: minimo
	
\item[f)]

\textbf{problema F}$([a : \mathbb{Z}])=r: \mathbb{Z}$\{ \\
	\textbf{  requiere:} $|a|>0;$\\
	\textbf{  asegura:}$(\exists i,j:\mathbb{Z})(0 \leq i \wedge i\leq j \wedge j<|a| \wedge todosIgualesEntreIndices(i,j,a) \wedge ((\forall l,m:\mathbb{Z})(0\leq l \wedge l\leq m \wedge m<|a| \wedge todosIgualesEntreIndices(l,m,a) \rightarrow j-i\geq m-l) \wedge  r==j-i);$\\
	\}
	
$todosIgualesEntreIndices(i,j:\mathbb{Z},[a:\mathbb{Z}])\equiv \{(\forall k:\mathbb{Z})(i\leq k \wedge k<j \rightarrow a[k]== a[i]) \} $

Nombre propuesto tamañoDeLaMayorSublistaTodosIguales
\end{enumerate}

\end{document}